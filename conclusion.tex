\chapter{Conclusion}
\label{cha:conc}

This thesis is aiming to make a better exploration of the G1 family of garbage collectors
by implementing and measure them carefully on JikesRVM.
The explorations and discussion in the previous chapters has successfully demostrates
the pros and cons of each algorithm and the cause of these phenomenons.

\section{Future Work}
\label{sec:future}

Although this thesis has come to an end, the research on this topic is far to finish.
There are still much work to do after this research project. In general, it includes:
$1.$ Resolve a data rece problem for Shenandoah GC.
$2.$ Perform some optimizations
$3.$ Exploring ZGC.
$4.$ Further G1 related GC research

\subsection{Race Problem for Shenandoah GC}

Currently the implementation of Shenandoah GC has a data race problem.
Based on the previous analysis, the mutator is not lock and release a monitor
expectedly during the concurrent phase of the Shenandoah GC, when there can be
two different copies of an object exist in the heap.

Duo to the time scope of thie thesis, I cannot solve this issue currently.
But currently a work-around is implemented in the Shenandoah GC, with a lower mutator performance.
This explains the bad performance of mutator overhead we evaluated in chapter~\ref{cha:evaluation}.
As the most urgent problem I are facing, I plan to solve this issue immediately after
this project.

\subsection{Optimizations}

As part of the mutator latency results discussed in chapter~\ref{cha:evaluation},
the generational G1 currently does not reveal any peprformance gain compared to
the non-generational G1. It is expected to have more optimizations and parameter
tuning to the generational G1 to further increase its performance.

I am also considering the possibility to make the implemented Garbage-first
and Shenandoah GC become production ready. One major part missing is the Optimizations
since the development of these garbage collectors was following the original design
of these collectors and did not have too much optimizations. After performing some
optimizations on these collectors as well as some additional correctness verfication,
the collector can have the possibility to become production ready.

\subsection{ZGC}

Due to the time scope of this project, I did not implement and measure ZGC.
In addition, since JikesRVM and MMTk only supports the 32bit address space but the pointer
coloring process in ZGC requires the 64bit address space, this makes the implementation
of ZGC more difficult.
But as a latest member of the Garbage-first family of collectors, ZGC is
worth for implementation and measurements.
However, a detailed plan of resolving several hardware incompactabilities should be done in the future,
before starting the implementation of ZGC.

\subsection{Future G1 related GC research}

\pending{???}


\section{Summary}

This thesis is aiming to explore the detail of different G1 family of garbage collectors
by implementing them as a series of refinements and measure them carefully.

As discussed in previous chapters, most of my implementations result in a reasonable
or even unexpectedly performance for either GC pause time and mutator overheads.
Based on these implementations and measurement results, the pros and cons as well
as their underlying reasons are explored and discussed. Hence, these explorations can
further inspire GC designers to reconsider the region-based GC algorithms to make
further improvements.

In conclusion, although the algorithms supporting the G1 family of garbage collectors
increase the GC performance significantly, there is still a lot to improve and more research
should be done in this area.

% Implelemtations I did

% How was the evaluation performed

% A little discussion of the evaluation results and further suggestions

% A little bit of future work

