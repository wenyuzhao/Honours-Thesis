\chapter{Conclusion}
\label{cha:conc}

This thesis is aiming to identify and explore the underlying relationship of the
G1 family of garbage collectors as well as analyze the performance impact caused by
each structural component.
The explorations and discussions in the previous chapters have successfully demonstrated
that the relationship among the G1 family of collectors exists and can impact the GC
performance in both positive and negative ways.
Hence the pros and cons of each algorithm and the cause of these phenomena
are discussed based on the evaluation results.

\section{Future Work}
\label{sec:future}

Although this thesis has come to an end, the research on this topic is far to finish.
There is still much work to do after this research project.
In general, a few typical future works are:
$1.$ Resolve a data race problem for Shenandoah GC.
$2.$ Perform some optimizations
$3.$ Exploring C4 GC and ZGC.
$4.$ Further G1 related GC research

\subsection{Race problem for Shenandoah GC}

Currently, the implementation of Shenandoah GC has a data race problem.
Based on the previous analysis, the mutator is not locking and releasing monitors
expectedly during the concurrent phase of the Shenandoah GC, when there can be
two different copies of an object exist in the heap.

Due to the time scope of this thesis, I cannot solve this issue currently.
But currently, a workaround is implemented in the Shenandoah GC, with a lower mutator performance.
This explains the bad performance of mutator overhead we evaluated in chapter~\ref{cha:evaluation}.
As the most urgent problem I am facing, I plan to solve this issue immediately after
this project.

\subsection{Optimizations}

As part of the mutator latency results discussed in chapter~\ref{cha:evaluation},
the generational G1 currently does not reveal to much performance gain compared to
the non-generational G1, expect the decrease of full GC ratio. It is expected to have more optimizations and parameter
tuning to the generational G1 to further increase its performance.

I am also considering the possibility to make the implemented Garbage-first
and Shenandoah GC become production ready. One major part missing is the optimization
since the development of these garbage collectors was following the original design
of these collectors and did not have too many optimizations. After performing some
optimizations on these collectors as well as some additional correctness verification,
the collector can have the possibility to become production ready.

\subsection{C4 GC and ZGC}

Due to the time scope of this project, I did not implement and measure C4 GC and ZGC.
In addition, since JikesRVM and MMTk only supports the 32bit address space but the pointer
coloring process in ZGC requires the 64bit address space, this makes the implementation
of ZGC more difficult.

However, as the latest member of the Garbage-first family of collectors, C4 and ZGC
generally have better performance than G1 GC and Shenandoah GC. By performing most of the
GC work concurrently, the pause time of C4 and ZGC are not proportional to the heap size and
are expected to be less than 10 milliseconds even targeting 100\,GB heaps accoridng to \cite{liden_karlsson_2018}.
In this way, these two collectors are extremely worth for an exploration.

However, a detailed plan for resolving several hardware incompatibilities should be done in the future,
before starting the implementation of these two collectors.

\subsection{Future G1 related GC research}

The Garbage-first family of garbage collectors have been proved to be high performance,
in terms of GC latency and mutator overheads. However, the measurement results are still
not perfect, which means there is a lot more can be done to further improve GC performance.

One example can be adding a generational extension to the Shenandoah GC to collect
young garbage as early as possible to reduce the frequency of falling to full GCs.
This generational mode involves a remembered set to remember object pointers in mature
space pointing to the nursery space. However, in addition to using the table-based remembered
sets used in G1, the buffer-based remembered sets introduced by \cite{blackburn2008immix} can
also be explored as a comparison to G1's remembered sets.

There are still many details of G1 related garbage collection algorithms can be
explored. By performing more improvements over the existing Garbage-first family of collectors
(e.g. the Shenandoah GC), GC performance can be expected to have more improvements.

\section{Summary}

This thesis is aiming to identify and explore the underlying relationship of the
G1 family of garbage collectors, implement them as a series of collectors to
reflect such relationships and analyze GC performance impact of different algorithm components.

As discussed in previous chapters, the potential relationships among the G1 family of collectors
are successfully discovered and discussed.
Based on such relationships, this thesis produces the first implementation of the
G1 family of collectors that reflect on the underlying algorithmic relationships.
Most of my implementations result in a reasonable
or even unexpectedly performance for either GC pause time and mutator overheads.
Based on these implementations and measurement results, the pros and cons, as well
as their underlying reasons, are explored and discussed.
Hence, the explorations performed in this thesis can inspire GC designers to
reconsider the design and structure of region-based GC algorithms to make
further valuable improvements and invent new algorithms.

In conclusion, the relationships among the G1 family of collectors exist and have
an impact on GC performance in many ways, including negative performance impacts.
Which mean there is still a lot to improve and more research should be done in this area.

% Implelemtations I did

% How was the evaluation performed

% A little discussion of the evaluation results and further suggestions

% A little bit of future work

