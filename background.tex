\chapter{Background and Related Work}
\label{cha:background}

This chapter describes the background and basic ideas of several garbage collectors,
particularly those targeting the java virtual machine and are implemented in OpenJDK or JikesRVM,
as well as the differences among several garbage collection techniques.

In addition, this chapter also performed a general discussion of the related work
on analyzing region-based garbage collectors.
\\\\
Section~\ref{sec:gcalgorithms} roughly discusses and compares different class of GC algorithms.
\\\\
Section~\ref{sec:g1collectors} describes the general design of the Garbage-first family of garbage collectors.
\\\\
Section~\ref{sec:relatedwork} describes the related work on analyzing region-based garbage collectors.
\\\\
\section{Categories of GC algorithms}
\label{sec:gcalgorithms}

This section discusses the main classes of garbage collection algorithms, as well
as their pros and cons.

\subsection{Reference counting}



\subsection{Mark \& sweep GC}



\subsection{Copying GC}



\section{Garbage-First family of garbage collectors}
\label{sec:g1collectors}

\subsection{Garbage-First GC}

\subsection{Shenandoah GC}



\section{Related work}
\label{sec:relatedwork}
You may reference other papers. For example: 
Generational garbage collection~\citep{LH:83,Moon:84,Ungar:84} is perhaps the
single most important advance in garbage collection since the first collectors
were developed in the early 1960s. (doi: "doi" should just be the doi part, not
the full URL, and it will be made to link to dx.doi.org and resolve.
shortname: gives an optional short name for a conference like PLDI '08.)



\section{Summary}

In this section we introduced the background and basic ideas of several garbage collectors and
also performed a general discussion of the related work on analyzing region-based garbage collectors.

Next chapter will briefly describe the implementation environment of the collectors implemented
in this thesis, including the high-level design of JikesRVM and MMTk.
