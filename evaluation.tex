\chapter{Performance Evaluation}
\label{cha:evaluation}

This chapter discusses the performance evaluation we undertaken for analyzing
the Garbage-First family of garbage collectors.
This chapter will firstly discusses the software and hardware platform we used for benchmarking,
Then the detailed evaluation process and resutls of both pause time and
barrier latency will be presented and discussed.


\section{The Dacapo Benchmark} % 2

We performed all the following pause time evacuations and barrier latency evaluation
by using the Dacapo Benchmark. The benchmarking suites we 

\section{Hardware Platform} % 2

During the implementation of all the G1 family of garbage collectors in chapter~\ref{cha:implementation},
a list of machine with a large varities on CPU types, clock, number of processors and
the size of cache and memory were involved, as shown in Table~\ref{tab:machines}.
By executing all the benchmarking suite of the Dacapo Benchmark on these different machines,
and thanks to the benchmarking suites of the Dacapo Benchmark which reflect
that different categories of programs in the real world,
we has the ability to statistically verify the correctness of the prevously
implemented garbage collectors (in chapter~\label{cha:implementation})
and make sure they performs as intended in a real world setting.

For further performance evaluation, we chose the "fisher" machine for benchmarking.

\begin{table*}
  \centering
  \label{tab:machines}
  \input table/machines.tex
  \caption{Machines used for development and evaluation.}
\end{table*}

\section{Pause Time Evaluation} % 7
\label{sec:pausetimeevaluation}
\subsection{MMTk harness callbacks}
\subsection{Mutator latency timer}
\subsection{Results}
\subsection{Discussion}


\section{Barrier Latency Evaluation} % 7
\label{sec:barrierlatencyevaluation}
\subsection{Methodology}
\subsection{Snapshot-at-the-begining barriers}
\subsection{Remembered set barriers}
\subsection{Brooks indirection pointer barriers}
\subsection{Discussion}

\section{Summary} % 2
\label{sec:summary}

%%% Local Variables: 
%%% mode: latex
%%% TeX-master: "paper"
%%% End: 


% We can also refer to specific lines of code in code listings. The bug in
% \Cref{fig:c:hello} is on \cref{line:bug}. There is also a bug in
% \Cref{fig:java:hello} on \crefrange{line:jbug-start}{line:jbug-end}. To
% achieve these references we put
% \texttt{(*@ \textbackslash label\{line:bug\} @*)}
% in the code -- the \texttt{(*@ @*)} are escape delimiters that allow you to add
% LaTeX in the (otherwise verbatim) code file.

% \begin{table*}
%   \centering
%   \caption{Processors used in our evaluation.  Note that the caption for a table is at the top.  Also note that a really long comment that wraps over the line ends up left-justified.}
%   \label{tab:machines}
%   \input table/machines.tex
% \end{table*}

% \begin{figure}
%   \centering
%   \begin{subfigure}[b]{\textwidth}
%       \lstinputlisting[linewidth=\textwidth,breaklines=true]{code/hello.c}
%       \caption{C}
%       \label{fig:c:hello}
%   \end{subfigure}

%   \begin{subfigure}[b]{\textwidth}
%       \lstinputlisting[linewidth=\textwidth,breaklines=true]{code/hello.java}
%       \caption{Java}
%       \label{fig:java:hello}
%   \end{subfigure}

%   \caption{Hello world in Java and C. This short caption is centered.}
%   \label{fig:helloworld}
% \end{figure}