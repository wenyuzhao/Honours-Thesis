\chapter*{Abstract}
\addcontentsline{toc}{chapter}{Abstract}
\vspace{-1em}

% Problem

Region-based garbage collectors, including Garbage-first GC, Shenandoah GC, C4 GC, and ZGC,
share a lot of common designs and algorithms, such as
the similar marking algorithm, memory structure, and allocation algorithm.
However, the existence of such underlying relationship among these collectors
has never been well identified and explored.
Instead, the designers of these region-based collectors tend to treat these as
stand-alone collectors instead of an improvement to the existing collector.
Such ignorance of the underlying relationships among the region-based collectors
can mislead the future design of related garbage collectors. 

Hence, analysis, measurements, and comparisons among these collectors can also be hard.
Although the design and algorithms of these collectors are similar,
their structural relationships are not reflected in the
original design and implementations of these collectors.
For this reason, no one has ever successfully measured
the GC performance contribution of some specific part of the GC algorithm or the extension component involved
in these collectors.
Which means no one can properly understand the pros and cons
of the design of these collectors, and may further lead to some potential performance
issue due to the inappropriate GC design.

% Contribution

As the fundamental novel contribution, this thesis is the first to identify the existence of a
structural relationship among the region-based collectors and makes a deep exploration
of the structural relationships among a set of region-based collectors which have a similar
design to the Garbage-First GC. In this thesis I use the term
"The Garbage-First Family of Garbage Collectors" to describe such category of collectors.

This thesis produces the first implementation to reflect the previously
discovered structural relationships.
Specifically, in this thesis, I discuss the implementation details of a total of six G1 family
of collectors, starting from a simple region-based
collector to the Garbage-first GC and Shenandoah GC.
Each collector is implemented as an improved version of the previous collector to reflected
the corresponding algorithmic relationship.

Based on such implementation, this thesis performs a detailed
and careful analysis of the GC performance contribution of each component of the algorithms.
This includes the measurement of the GC pause time, several mutator barrier overheads
and the space overhead of remembered-sets.

% Results

The exploration of the Garbage-First Family of Garbage Collectors leads to a conclusion
that there exists a structural relationship among these G1 family of collectors.
Instead of being stand-alone collectors, they tend to be collectors with
algorithmic improvements to some existing collectors.

As the result of the GC performance evaluation,
structural components including concurrent-marking, remembered-sets and concurrent
evacuation contributes to an improvement of 88.5\%, 10.7\% and 72.6\%
respectively to the average GC pause time on the DaCapo benchmark suite.
By using remembered-sets, G1 has 8.36\% average footprint overhead.
Mutator barriers performance SATB barrier, remembered-set barrier,
and Brooks barrier increases the mutator overheads by 22.0\%, 59.6\%, and 85.46\%
respectively.

% Meaning

The explored algorithmic relationships among the G1 family of collectors can help
GC designers or other programmers working with JVM to have a deeper understanding of
the G1 family of collectors as well as their relationships and the pros and cons of their involved improvements.
The measured GC performance contribution of each component of the GC algorithm
can help garbage collection algorithm designers to reconsider the design of 
the region-based garbage collectors and memory structures, identify
the main advantages and drawbacks of each involved GC algorithm and
hence have the ability to make further optimizations to them.

%%% Local Variables: 
%%% mode: latex
%%% TeX-master: "paper"
%%% End: 