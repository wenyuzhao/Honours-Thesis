\chapter*{Abstract}
\addcontentsline{toc}{chapter}{Abstract}
\vspace{-1em}

% Problem

The design and implementation of different region-based collectors has been well explored, from the
most simple Mark-Region collector created by \cite{sergeant2014improving} to the
more complex Garbage-first GC, Shenandoah GC, C4 GC and ZGC created by
\citep{detlefs2004garbage,flood2016shenandoah,tene2011c4,liden_karlsson_2018} respectively.
Most of these collectors have been proved to have the ability to
significantly improve the GC performance.
Particularly, Garbage-first GC has been widely used in modern industry to reduce
GC latency on large heaps.

These region-based collectors share a lot of common designs and algorithms, such as
the similar marking algorithm, memory structure and allocation algorithm.
However, the existence of such underlying relationship among these collectors
has never been well identified and explored.
Instead, the designers of these region-based collectors tend to treat these as
stand-alone collectors instead of a refinement to the existing collector.
Such ignorance of the underlying relationships among the region-based collectors
can mislead the future design and refinements on related collectors. 

Hence, analysis, measurements and comparisons among these collectors can also be hard.
Although the design and algorithms of these collectors are similar,
their structural relationships are not reflected to the
original design and implementations of these collectors.
For this reason no one have ever succsessfully measured
the GC performance contribution of some specific GC algorithm or refinement component involved
in these collectors.
Which means no one can properly understand the pros and cons
of the design of these collectors, and may further leads to some potential performance
issue due to inappropriate GC design.

% Contribution

As the fundamental novel contribution, this thesis is the first to prove the existence of a potential
structural relationship among the region-based collectors and makes a deep exploration
of the refinements relationships among a set of region-based collectors which have similar
design to the Garbage-First GC. In thesis I use the term
"The Garbage-First Family of Garbage Collectors" to describe such category of collectors.

This thesis produces the first implementation to reflect the previously
discovered refinements relationships.
Sperifically, in this thesis, I discuss the implementation details of a total of six G1 family
of collectors, starting from a simple region-based
collector to the Garbage-first GC and Shenandoah GC.
Each collector are implemented as a refined version of the previous collector to reflected
the corresponding refinement relationship.

Based on such implementation, this thesis performs a detailed
and careful analysis of the GC performance contribution by each part of the refinement.
This includes the measurement of the GC pause time, several mutator barrier overheads
and the space overhead of remembered-sets.

% Results

The exploration of the Garbage-First Family of Garbage Collectors leads to a conclusion
that there exist a refinement relationship among these G1 family of collectors.
Instead of being stand-alone collector, they tend to be collectors with
algorithmic refinements to some existing collectors.

As the result of the GC performance evaluation,
refinement components including concurrent-marking, remembered-sets and concurrent
evacuation contributes to a improvement of \pending{???}\%, \pending{???}\% and \pending{???}\%
respectively to the average GC pause time on the DaCapo benchmark suite.
By using remembered-sets, G1 has \pending{???}\% average footprint overhead.
Mutator barriers performance SATB barrier, remembered-set barrier,
and Brooks barrier increases the mutator overheads by \pending{???}\%, \pending{???}\%
and  \pending{???}\%
respectively.

% Meaning

The explored algorithmic relationships among the G1 family of collectors can help
GC designers or other programmers working with JVM to have a deeper understanding of
the G1 family of collectors as well as their relationships and the pros and cons of their involved refinements.
The measured GC performance contribution of each refinement of the GC algorithm
can help garbage collection algorithm designers to reconsider the design of 
the region-based garbage collectors and memory structures, identify
the main advantages and drawbacks of each involved GC algorithm and
hence have the ability to make further refinements and optimizations to them.

%%% Local Variables: 
%%% mode: latex
%%% TeX-master: "paper"
%%% End: 