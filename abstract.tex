\chapter*{Abstract}
\addcontentsline{toc}{chapter}{Abstract}
\vspace{-1em}

The Garbage-First family of garbage collectors, including Garbage-first GC,
Shenandoah GC and ZGC, are region-based concurrent collectors designed for
large heaps to archive lower latency and mutator overhead.
Particularly, Garbage-first GC is widely used in modern industry.
However, analysis, measurements and comparisons between their OpenJDK implementations can be
hard since the implementations do not share too much code and have different
level of optimizations.

This thesis attempts to perform a detailed and
careful analysis of the performance of these collectors at an algorithmic level
by implementing them JikesRVM and share as much design as possible.

In this thesis, I will discuss implementation details of starting from a simple region-based
collector and performs progressive refinements to further build the Garbage-first GC
and the Shenandoah GC. At each progressive refinement step I will discuss the detailed
design of the refinement, the barriers or data structures involved and the pros and cons
of involving such refinement.

As the part of performance evaluation, I evaluated the footprint overhead,
the mutator latency and the mutator barrier overheads for these collectors.
My results show that by using remembered-sets, G1 has \pending{???}\% average footprint overhead
and G1, Shenandoah has barrier overheads of \pending{???}\% and \pending{???}\%
respectively on the DaCapo benchmark suite. I also find that by using Brooks
style indirection pointers, Shenandoah and ZGC has average barrier overheads
of \pending{???}\% and \pending{???}\%. The costs and overheads revealed in these
garbage collectors can help garbage collection algorithm designers to identify
the main advantages and drawbacks of region-based concurrent collectors and
hence have the ability to make further refinements and optimizations to them.

In additon, the refinement steps discussed in this thesis reveal the relationship
among the Garbage-First family of garbage collectors, which can help garbage collection algorithm designers
to reconsider the design of garbage collectors and memory structures to make further algorithmic improvements.

%%% Local Variables: 
%%% mode: latex
%%% TeX-master: "paper"
%%% End: 