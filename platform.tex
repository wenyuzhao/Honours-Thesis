\chapter{JikesRVM and MMTk}
\label{cha:platform}

This chapter describes the general design and structure of the platform where the all Garbage-First
family of garbage collectors were implemented and evaluated. This includes JikesRVM
and MMTk.
\\\\
Section~\ref{sec:jikesrvm} describes the design of JikesRVM, especially the Java object memory model.
\\\\
Section~\ref{sec:mmtk} describes the design of MMTk, as well as its important structures.
\\\\

\section{JikesRVM}
\label{sec:jikesrvm}

JikesRVM is a research purpose Java virtual machine. It is a meta-circular JVM
which is implemented in the Java programming language and is self hosted.
JikesRVM was designed to provide a flexible open sourced test-bed to
experiment with virtual machine related algorithms and technologies.

In stead of executing Java programs by directly interpreting the Java byte code,
JikesRVM compiles them into machine code and executes them.
During the build time, JikesRVM requires a hosting JVM to run the boot image compiler
to compile the whole JikesRVM into machine code.
Then during running time, when loading Java class files and invoking java methods,
JikesRVM will compile the related Java byte code into corresponding machine code and execute it.

JikesRVM implemented two tiers of compilers, the baseline compiler and the optimizing compiler.
The baseline compiler simply translates the Java byte code into machine code and do no
optimizations while the optimizing compiler performs several optimizations during the
code generation phase. The targeting machine of JikesRVM includes Inter x86 computers
and the IBM PowerPC.

\subsection{Object model}

The core part of the memory structure of JikesRVM is the Java object memory model.



\section{MMTk}
\label{sec:mmtk}

\subsection{Plan, collectors and mutators}

\subsection{Space and allocators}

\section{Summary}

In this section I discussed the general design and structure of JikesRVM and MMTk.
Based on the MMTk and the design of Java object memory model, the next chapter will
describe the implementation detail of all the Garbage-First family of garbage collectors
discussed in this thesis.
